\documentclass[twoside]{article}
\usepackage{ukai}

\title{\customtitle{Your Paper Title}}

\setauthorsshort{Author One et al.}


\author{
    Author One$^1$, Author Two$^2$, Author Three$^1$ \\
    {$^1$Affiliation One, $^2$Affiliation Two} \\
    \texttt{author.one@example.com, author.two@example.com, author.three@example.com}
}

\date{}

\begin{document}
\maketitle

\begin{abstract}
This is the abstract. It summarises the paper's key contributions in a maximum of 200 words.
\end{abstract}

\keywords{keyword 1, keyword 2, keyword 3}

\section{Introduction}

Introduce the content of the paper here. Cite references using \cite{example}. The following sections are samples of how to include different content in the paper. You can customise these sections with your paper sections and content. For more, see LaTeX documentation \footnote{LaTeX Project - \url{https://www.latex-project.org}}.


\section{Approach}

\subsection{Mathematical Notation}

Equations can be written using the \texttt{align} environment for multi-line equations:

\begin{align}
    E &= mc^2 \\
    F &= ma
\end{align}

Or the \texttt{equation} environment for single-line equations:

\begin{equation}
    \nabla \cdot \vec{E} = \frac{\rho}{\varepsilon_0}
\end{equation}

\section{Figures}

Include figures using \texttt{graphicx}:

\begin{figure}[ht]
    \centering
    \includegraphics[width=0.7\textwidth]{example-image}
    \caption{This is a sample figure. Replace \texttt{example-image} with your file.}
    \label{fig:example}
\end{figure}

\section{Algorithms}

Pseudocode can be included using the \texttt{algorithm} and \texttt{algorithmic} packages:

\begin{algorithm}
\caption{Sample Algorithm}
\label{alg:example}
\begin{algorithmic}[1]
\STATE Initialize $x \gets 0$
\FOR{$i \gets 1$ to $n$}
    \STATE $x \gets x + i$
\ENDFOR
\RETURN $x$
\end{algorithmic}
\end{algorithm}

\section{Conclusions}

Summarize the contributions of your paper.

\bibliographystyle{ieeetr}
\bibliography{references}

\end{document}
